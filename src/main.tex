%Copyright 2014 Jean-Philippe Eisenbarth
%This program is free software: you can 
%redistribute it and/or modify it under the terms of the GNU General Public 
%License as published by the Free Software Foundation, either version 3 of the 
%License, or (at your option) any later version.
%This program is distributed in the hope that it will be useful,but WITHOUT ANY 
%WARRANTY; without even the implied warranty of MERCHANTABILITY or FITNESS FOR A 
%PARTICULAR PURPOSE. See the GNU General Public License for more details.
%You should have received a copy of the GNU General Public License along with 
%this program.  If not, see <http://www.gnu.org/licenses/>.

%Based on the code of Yiannis Lazarides
%http://tex.stackexchange.com/questions/42602/software-requirements-specification-with-latex
%http://tex.stackexchange.com/users/963/yiannis-lazarides
%Also based on the template of Karl E. Wiegers
%http://www.se.rit.edu/~emad/teaching/slides/srs_template_sep14.pdf
%http://karlwiegers.com
\documentclass{scrreprt}

\usepackage{underscore}
\usepackage[bookmarks=true]{hyperref}
\usepackage[utf8]{inputenc}
\usepackage[toc,acronym,section=section]{glossaries}
\usepackage[french]{babel}
\usepackage{graphicx}
\usepackage[T1]{fontenc}
\usepackage{listings}

% to suppress glossaries page break

\hypersetup{
    bookmarks=false,    % show bookmarks bar?
    pdftitle={Cahier des Charges},    % title
    pdfauthor={Jean-Philippe Eisenbarth},                     % author
    pdfsubject={TeX and LaTeX},                        % subject of the document
    pdfkeywords={TeX, LaTeX, graphics, images}, % list of keywords
    colorlinks=true,       % false: boxed links; true: colored links
    linkcolor=blue,       % color of internal links
    citecolor=black,       % color of links to bibliography
    filecolor=black,        % color of file links
    urlcolor=blue,        % color of external links
    linktoc=page            % only page is linked
}%
\def\myversion{0.1 }
\date{}
%\title
\usepackage{hyperref}
\newcounter{reqfcounter}
\newcounter{reqnfcounter}


\newenvironment{reqf}[2]{
\refstepcounter{reqfcounter}
\textbf{Requirement \thereqfcounter} \space\\
Description : #1 \\
Priorité : \textbf{#2}\\
\textit{Détails, contraintes et conséquences}
\begin{enumerate}}{\end{enumerate}\vspace{1em}}


\newenvironment{reqs}[1]{
\refstepcounter{reqnfcounter}
\textbf{Requirement \thereqnfcounter} \space\\
Description : #1\\
}

\makeglossaries

\newglossaryentry{latex}
{
        name=latex,
        description={Is a mark up language specially suited for 
scientific documents}
}
\newglossaryentry{livret}
{
        name={Livret de l'élève},
        description={Il contient toutes les informations concernant la scolarité des étudiants}
}

\begin{document}
\renewcommand*{\glsclearpage}{}
\begin{center}
    \rule{16cm}{5pt}\vskip1cm
    \begin{bfseries}
        \Huge{Cahier des charges}\\
        \vspace{1.9cm}
        Le projet de MOCI\\
        \vspace{1.9cm}
        \LARGE{Version \myversion}\\
        \vspace{1.9cm}
        Préparé par Amine Salhi, Florian Touraine, Ewan Decima, Hippolyte Cosserat\\
        \vspace{1.9cm}
        Télécom Nancy\\
        \vspace{1.9cm}
        \today\\
    \end{bfseries}
\end{center}

\tableofcontents

\chapter*{Historique}

\begin{center}
    \begin{tabular}{|c|c|c|c|}
        \hline
	    Name & Date & Reason for changes & Version\\
        \hline
	    Francois Charoy & 29/06/2020 & initial & 0.1\\
        \hline
	    Francois Charoy & 29/10/2020 & initial & 0.1\\
        \hline
	    Martine Gautier & 30/03/2021 & orthographe & 0.1.1\\
        \hline
        Brigitte Wrobel-Dautcourt & 22/05/2021 & ponctuation, typo, orthographe, grammaire... & 0.1.2\\
        \hline
        Martine Gautier & 07/07/2021 & ponctuation, typo, orthographe, grammaire, ... & 0.1.3\\
        \hline
    \end{tabular}
\end{center}

\chapter{Introduction}
Ce cahier des charges est réalisé dans le cadre du cours de TELECOM Nancy  et fait référence aux livres suivants~\cite{Sommerville:2010:SE:1841764,Pohl:2010:REF:1869735,Rumbaugh:2004:UML:993859}.

\section{Objectif}
Le livrable décrit dans ce cahier des charges est un système de gestion des recrutements pour la PME RecrutExpress. L'objectif du système est de simplifier et d'optimiser le processus de recrutement de l'entreprise. Il couvre différentes étapes : de la création des offres d'emploi à l'intégration d'employés. À cet effet, le cahier des charges concerne le système complet et non une partie isolée. Il doit permettre la publication des offres, la gestion des candidatures, la planification des entretiens, la communication avec les candidats, et l'analyse des métriques de recrutement, tout en s'intégrant aux systèmes existants de l'entreprise.

\section{Audience et sections importantes}
Les lecteurs potentiels de ce document sont les développeurs d'une part, la direction de la PME, qui doit pouvoir contrôler que le système fonctionnera bien selon ses attentes.

\section{Portée du projet}
Le système est en charge de la gestion du processus de recrutement. Son but est de centraliser et d'automatiser les tâches liées au recrutement, de la publication des offres à l'intégration des nouveaux employés. La saisie de nouveaux formulaire de candidature doit être simplifié au maximum afin de faciliter le travail des ressources humaines.

Le processus de candidature doit être simplifié au maximum, pour attirer plus de candidats qualifiés et faciliter le travail des recruteurs. 

Il doit permettre au service des ressources humaines ainsi qu'à direction un suivi plus efficace des candidatures et des métriques de recrutement, permettant une prise de décision plus rapide et éclairée.


\section{Conventions}
$<$Décrivez les conventions de nommage et de structure utilisées au long du document ; indiquez de quelle façon ces conventions peuvent aider le lecteur.$>$

\section{Références}

Le Fascicule 0 contient les règles de la gestion des absences.
\url{https://intranext.telecomnancy.univ-lorraine.fr/xwiki/bin/download/FISE/WebHome/FASCICULE-0A_2019-2020\%20_FISE.pdf}

\chapter{Description}
\section{Perspective du produit}
$<$Indiquez le contexte et l'origine du produit ainsi que les fonctionnalités attendues. $>$

\section{Principales fonctionnalités}

\subsection{Gestion des candidatures}
\begin{itemize}
    \item Gestion totale des formulaires de candidature
    \item Système de proposition de formulaire de candidature
    \item Consultation, recherche et filtrage facile les profils et candidatures
    \item Sauvegarde des décisions
    \item Système de préselection soumis à l'équipe de direction
    \item Utilisation de modèles de mails pré-écrits et envois automatiques, gestion facile des rendez-vous avec tous les partis concernés
\end{itemize}

\subsection{Administration du système}
\begin{itemize}
    \item Création et gestion de rôles pour faciliter le recrutement
    \item Tableau de bord du système pour accéder aux informations les plus essentielles
    \item Suivi et gestion de métriques
\end{itemize}

\subsection{Capacité d'intégration}
\begin{itemize}
    \item Le logiciel s’intégre dans une hiérarchie LDAP et avec les systèmes internes de gestion de paye et ERP pour spécifier l’embauche
    \item Multiples intégrations avec des sites d'annonces d'emplois, 
\end{itemize}

\subsection{Sécurité et confidentialité}
\begin{itemize}
    \item Système sécurisé par SSO
    \item Journaux et historiques d'actions sur les données privées
    \item Logiciel conforme à la RGPD
\end{itemize}

\subsection{Interface pour les candidats}
\begin{itemize}
    \item Authentification rapide
    \item Consultation des informations transmises et suivi candidatures
    \item Contrôle total des données personnelles
    \item Retour sur l'expérience de canditure
\end{itemize}

\section{Description des utilisateurs}

Le système de gestion des recrutements de RecrutExpress est conçu pour répondre aux besoins de plusieurs types d'utilisateurs au sein de l'organisation. Voici les principaux groupes d'utilisateurs identifiés :

\subsection{Service Ressources Humaines (RH)}
\begin{itemize}
    \item Le service RH est responsable de la gestion quotidienne du processus de recrutement.

    \item Le service RH est en charge de la planification de rendez vous suite à une candidature.

    \item Le service RH  conceptualise les offres d'emploi et formulaires de candidature associé

    \item Le service RH rédige les modèles de mails automatiques : accusé réception de rendez vous, refus ou validation d'embauche ...

    \item Le service RH est en charge de la génération de métriques
    
     
\end{itemize}

\subsection{Candidats}
\begin{itemize}
    \item Les candidats externes utilisent le système pour postuler et suivre leurs candidatures.
    \item  Les employés peuvent postuler à de nouveaux postes au sein de l'entreprise.
\end{itemize}

\subsection{Administrateurs Système}
\begin{itemize}
    \item Les administrateurs système gèrent les accès utilisateurs.
    \item Les administrateurs système sont en charge des protocoles sécurité.  
    \item Les administrateurs système supervise la maintenance du système.
    \item Les administrateurs système assurent la conformité RGPD et la sécurité des données.
    \item Les administrateurs système sont responsables de l'intégration du système avec d'autres outiles de l'entreprise (Google Workspace, ERP ...).
    \item Les administrateurs système sont responsables de la création et de l’attribution des rôles, lesquels définissent les privilèges au sein du système d’information.
    \item Après formation, les administrateurs système assistent les différents utilisateurs internes en cas de problème technique.

\end{itemize}



\subsection{Direction}
\begin{itemize}
    \item Le service de direction approuve, ou nrefuse, les candidatures, la validation des besoins en recrutement et les décisions d'embauche.
    \item Surveille les indicateurs clés de performance (KPI) liés au recrutement.
\end{itemize}

\subsection{Services de Production}
\begin{itemize}
    \item Définissent les besoins en personnel et participent au processus de sélection.
    \item Peuvent être impliqués dans les entretiens et l'évaluation des compétences techniques.
\end{itemize}

\subsection{Service Comptable}
\begin{itemize}
    \item  Ont besoin d'accéder aux informations des nouveaux employés pour la gestion de la paie.
\end{itemize}



Chaque groupe d'utilisateurs aura des besoins spécifiques et des niveaux d'accès différents au sein du système, ce qui devra être pris en compte dans la conception et le développement de l'application.

\section{Environnement opérationnel}
$<$Expliquez où et comment va s'exécuter le logiciel : matériel, système et autres composants logiciels qui seront utilisés $>$

\section{Contraintes de conception et d'implantation}
$<$Décrire les contraintes qui vont impacter le développement: langage, sécurité, déploiement$>$

\section{Documentation}
$<$Décrire le contenu, le mode de fourniture et les standards. $>$

\section{Hypothèses/Dépendances}
$<$Détailler toutes les hypothèses qui peuvent potentiellement impacter la spécification technique, incluant des facteurs externes $>$

% \section{Contexte et origine}
% $<$Décrivez le contexte dans lequel va fonctionner le logiciel et éventuellement ce qu'il va remplacer ou compléter. Décrivez également rapidement les relations avec les autres systèmes de l'environnement.$>$
% \newacronym{ade}{ADE}{ADE est le logiciel d'emploi du temps de l'université de Lorraine}
% \newacronym{API}{API}{Application Programming Interface}

% L'emploi du temps des étudiants est accessible grâce aux \acrshort{API} d'\acrshort{ade}.

% Les informations concernant les règles d'absence se trouvent dans le \Gls{livret}.

% \section{Principales fonctionnalités}
% 

% \section{Les acteurs}
% $<$Décrivez les différents acteurs concernés par le système aussi bien pour l'utilisation que pour l'exploitation.$>$

% \section{Environnement opérationnel}
% $<$Décrivez le contexte dans lequel le logiciel va s'exécuter, où se trouvent les serveurs, qui va les opérer.$>$

% \section{Contraintes d'implantation et de conception}
% $<$Décrivez ce qui peut avoir un impact sur la mise en {\oe}uvre, comme des questions de réglementation, de support d'exécution, de limites techniques, d'outils à utiliser, de langage, de système.$>$

% \section{Hypothèses et dépendances}

% $<$Quelles sont les dépendances et les hypothèses qui vont orienter la construction du logiciel ?$>$


\chapter{Besoins fonctionnels}
$<$Pour chaque besoin fonctionnel, donnez une description détaillée de son fonctionnement permettant de comprendre plus précisément son fonctionnement. La numérotation doit correspondre aux besoins utilisateurs.$>$
~\\

\begin{reqf}{Le système doit permettre de saisir l'absence d'un étudiant à un cours.}{high}
\item L'absence est ajoutée aux absences de l'étudiant
\end{reqf}

\begin{reqf}{Le système doit permettre à un enseignant et à la direction des études d'éditer la liste des absences pour un étudiant.}{high}
\item L'utilisateur peut rechercher un étudiant à partir de son nom ou de son numéro INE. 
\item La recherche est facilitée par un mécanisme d'auto-complétion. 
\item La liste des absences s'affiche à l'écran : date, motif, justification.
\item Le total des absences justifiées et non justifiées est également affiché.
\end{reqf}

\begin{reqf}{Le système doit notifier l'étudiant lorsqu'une nouvelle absence le concernant est ajoutée.}{low}
\item Chaque fois qu'une absence est ajoutée par un enseignant ou par la scolarité, l'étudiant reçoit un message le prévenant 
\item Le message contient la date et l'heure de l'absence ainsi que le nombre d'absences déjà enregistrée 
\item Un étudiant peut choisir de ne plus recevoir de notifications.
\end{reqf}

\chapter{Besoin des interfaces externes}
\section{Interfaces Utilisateurs}
$<$Décrivez les interfaces utilisateur principales en incluant éventuellement des guides de conception, des contraintes et quelques exemples d'écrans si nécessaire.$>$

\section{Interfaces Matérielles}
$<$Décrivez les produits et les caractéristiques des appareils utilisés pour l'application ainsi que les interfaces de communication.$>$

\section{Interfaces Logicielles}
$<$
Décrivez les logiciels ou applications, les composants ainsi que leur version, les bases de données et les outils qui seront utilisés avec l'application, quels sont les protocoles utilisés et les données échangées.
$>$

\section{Interfaces de communication}
$<$
Décrivez interfaces et les standards de communication utilisés ainsi que les standards de sécurité utilisés pour les échanges de données et le chiffrage si nécessaire.
$>$
\chapter{Besoins non fonctionnels}

\section{Performance}
\begin{reqs}{Le temps de réponse pour les actions de l'utilisateur doit être inférieur à 100ms dans 99\% des cas.}
Le système doit être réactif et ne pas ralentir le travail de l'utilisateur. Des tests de performance seront mis en place pour vérifier ce besoin ainsi qu'une solution de monitoring pour vérifier que la performance attendue est bien atteinte.
\end{reqs}

\section{Sureté}
\begin{reqs}{Le système doit pouvoir être mis à jour sans être interrompu}
Les mises à jour logicielles doivent pouvoir être effectuées sans interruption du service.
\end{reqs}

\section{Sécurité}
\begin{reqs}{Le système doit fournir un moyen d'authentification à 2 facteurs}
\end{reqs}

\section{Qualité logicielle}

\section{Besoins liés au domaine}




\chapter{Appendices}
\section{Appendice A: Glossaire}
%see https://en.wikibooks.org/wiki/LaTeX/Glossary
\printglossaries

\section{Appendice B: Modèles d'analyse}

\subsubsection{Modèle de données}
$<$Diagramme entité association ou diagramme de classes correspondant aux informations nécessaires à l'application.$>$



\bibliographystyle{plain}
\bibliography{biblio}

\end{document}
